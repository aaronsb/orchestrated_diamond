% Options for packages loaded elsewhere
\PassOptionsToPackage{unicode}{hyperref}
\PassOptionsToPackage{hyphens}{url}
%
\documentclass[
  12pt,
  a4paper,
  bibliography=totoc,
  numbers=noenddot
]{scrartcl}
\usepackage{amsmath,amssymb}
\usepackage[backend=biber,style=numeric,citestyle=numeric,natbib=true]{biblatex}
\usepackage{csquotes}
\addbibresource{references.bib}
\usepackage{authblk}
\usepackage{setspace}
\usepackage{geometry}

% Page geometry
\geometry{
  a4paper,
  left=2.5cm,
  right=2.5cm,
  top=2.5cm,
  bottom=2.5cm
}

% Line spacing
\onehalfspacing
\usepackage{iftex}
\ifPDFTeX
  \usepackage[T1]{fontenc}
  \usepackage[utf8]{inputenc}
  \usepackage{textcomp} % provide euro and other symbols
\else % if luatex or xetex
  \usepackage{unicode-math} % this also loads fontspec
  \defaultfontfeatures{Scale=MatchLowercase}
  \defaultfontfeatures[\rmfamily]{Ligatures=TeX,Scale=1}
\fi
\usepackage{lmodern}
\ifPDFTeX\else
  % xetex/luatex font selection
\fi
% Use upquote if available, for straight quotes in verbatim environments
\IfFileExists{upquote.sty}{\usepackage{upquote}}{}
\IfFileExists{microtype.sty}{% use microtype if available
  \usepackage[]{microtype}
  \UseMicrotypeSet[protrusion]{basicmath} % disable protrusion for tt fonts
}{}
\makeatletter
\@ifundefined{KOMAClassName}{% if non-KOMA class
  \IfFileExists{parskip.sty}{%
    \usepackage{parskip}
  }{% else
    \setlength{\parindent}{0pt}
    \setlength{\parskip}{6pt plus 2pt minus 1pt}}
}{% if KOMA class
  \KOMAoptions{parskip=half}}
\makeatother
\usepackage{xcolor}
\usepackage{longtable,booktabs,array}
\usepackage{calc} % for calculating minipage widths
% Correct order of tables after \paragraph or \subparagraph
\usepackage{etoolbox}
\makeatletter
\patchcmd\longtable{\par}{\if@noskipsec\mbox{}\fi\par}{}{}
\makeatother
% Allow footnotes in longtable head/foot
\IfFileExists{footnotehyper.sty}{\usepackage{footnotehyper}}{\usepackage{footnote}}
\makesavenoteenv{longtable}
\ifLuaTeX
  \usepackage{luacolor}
  \usepackage[soul]{lua-ul}
\else
  \usepackage{soul}
\fi
\setlength{\emergencystretch}{3em} % prevent overfull lines
\providecommand{\tightlist}{%
  \setlength{\itemsep}{0pt}\setlength{\parskip}{0pt}}
\setcounter{secnumdepth}{3} % enable section numbering
\ifLuaTeX
  \usepackage{selnolig}  % disable illegal ligatures
\fi
\usepackage{bookmark}
\IfFileExists{xurl.sty}{\usepackage{xurl}}{} % add URL line breaks if available
\urlstyle{same}
\hypersetup{
  hidelinks,
  pdfcreator={LaTeX via pandoc}}

% Document metadata
\title{The Orchestrated Diamond: A Framework for Mixed-Initiative Co-Creation in Design}
\author{Aaron Bockelie}
\affil{Independent Researcher, Wichita, Kansas}
\date{\today}

\begin{document}

\maketitle

\begin{abstract}

The Double Diamond (DD) design process, while a seminal contribution to
design methodology, is increasingly ill-equipped for the speed,
complexity, and data-rich environment of modern product development.
This paper posits that the integration of Artificial Intelligence (AI)
can address these limitations. The central contribution of this research
is the proposal of a novel framework, The Orchestrated Diamond, which
evolves the DD by integrating a dynamic spectrum of human-AI
collaboration models. This framework maps specific interaction
paradigms---AI-in-the-Loop (AITL), Human-Led Synthesis, Mixed-Initiative
Co-Creation, and Human-in-the-Loop (HITL)---to the DD\textquotesingle s
four phases: Discover, Define, Develop, and Deliver. This structured
approach moves beyond informal metaphors for AI collaboration to offer a
robust, efficient, and innovative methodology for design. The
framework\textquotesingle s validity is established through a detailed
analysis of its benefits in overcoming the DD\textquotesingle s core
weaknesses, a frank assessment of its inherent challenges such as
algorithmic bias and the need for new designer competencies, and by
drawing parallels with empirical case studies from product development,
service design, and architecture. The paper concludes that The
Orchestrated Diamond offers a necessary evolution of design practice,
reframing the designer\textquotesingle s role from that of a sole
creator to a strategic orchestrator of a human-AI team. This has
profound implications for the future of design as a collaborative,
orchestrated process.
\end{abstract}

\textbf{Keywords:} Design methodology, Human-AI collaboration, Double Diamond, Mixed-initiative co-creation, Design process evolution, Artificial intelligence in design

\tableofcontents
\newpage

\section{Introduction: The
Confluence of Design Process and Artificial
Intelligence}\label{section-1-introduction-the-confluence-of-design-process-and-artificial-intelligence}

\subsection{The Imperative for Evolved
Design
Methodologies}\label{the-imperative-for-evolved-design-methodologies}

The contemporary enterprise landscape is undergoing a tectonic shift,
driven by the integration of Artificial Intelligence (AI) and automation
into the very fabric of operational workflows.\cite{cogleus2025rise} This
transition, termed "Automation 5.0" or the era of intelligent
automation, moves beyond simple task execution to embrace systems
capable of predictive decision-making, continuous self-learning, and
sophisticated cognitive functions.\cite{researchgate2025evolution} This evolution is
not merely incremental; it represents a fundamental rewiring of how
organizations run and create value.\cite{mckinsey2025state} Enterprise
systems, once static backbones, are now dynamic engines of insight and
efficiency, leveraging AI for everything from supply chain optimization
and personalized customer relationship management to automated HR
processes and predictive maintenance.\cite{cogleus2025rise}

Industry analysts have underscored the transformative scale of this
change. Gartner, for instance, identifies generative AI as a
general-purpose technology with a projected impact comparable to the
steam engine, electricity, and the internet.\cite{gartner2025generative} This
technology is poised to augment or automate core processes across a vast
array of industries, including pharmaceutical discovery, manufacturing,
media, architecture, and engineering.\cite{gartner2025generative} Forrester
reports that agentic AI---systems capable of setting goals, making
decisions, and taking action---is evolving generative AI from simply
producing "words" to executing "actions," fundamentally changing how
work gets done.\cite{nice2025forrester} As organizations increasingly embed
these intelligent capabilities into their operations, the methodologies
used to design the products, services, and systems that house them must
necessarily evolve in parallel. Traditional design frameworks, conceived
in a pre-AI era, risk becoming anachronistic, unable to fully harness
the potential or mitigate the risks of this new technological paradigm.

\subsection{The Paradox: Seminal
Frameworks and New
Paradigms}\label{the-paradox-seminal-frameworks-and-new-paradigms}

At the heart of modern design practice lies the Double Diamond (DD), a
process model popularized by the British Design Council in the
mid-2000s.\cite{wikipedia2025double1} It has served as a foundational framework
for a generation of designers, providing a simple, memorable, and
visually intuitive map of the design journey \cite{designcouncil2025history}. Its
core principle---a structured sequence of divergent and convergent
thinking across four phases (Discover, Define, Develop, Deliver)---has
offered a common language for designers to articulate and contextualize
their work for clients and stakeholders \cite{wikipedia2025double2}.

Concurrently, as AI tools have become more accessible, practitioners
across various fields have developed informal, grassroots models to
describe their new collaborative workflows. One of the most prevalent is
the "AI Sandwich" \cite{nationaljurist2025ai}. This simple but powerful metaphor
describes a Human-AI-Human interaction pattern: a human initiates a task
with a specific intent (the top slice of bread), an AI system generates
content, analysis, or options (the filling), and a human provides the
critical review, refinement, and final judgment (the bottom
slice) \cite{nationaljurist2025ai}. This model has found resonance in fields as
diverse as legal education, where students are taught to bracket
AI-generated content with their own analysis \cite{nationaljurist2025ai}, and
creative writing, where the human acts as the essential element between
AI-driven perception and action.\cite{medium2025symbiotic}

The coexistence of the formal, top-down Double Diamond and the informal,
bottom-up "AI Sandwich" reveals a significant paradox and a chasm
between established design theory and emergent AI practice. While
practitioners are creating ad-hoc mental models to cope with powerful
new tools, the formal methodologies they are trained in often fail to
account for this new reality. This paper addresses the central problem
arising from this disconnect: the venerable but increasingly challenged
Double Diamond process is ill-equipped to manage the speed, complexity,
and data-rich environment of modern development, while the informal
models of AI collaboration lack the structure and theoretical rigor
needed for systematic and scalable implementation.

\subsection{Thesis Statement and Research
Trajectory}\label{thesis-statement-and-research-trajectory}

This paper posits that the documented limitations of the Double Diamond
can be systematically addressed by moving beyond informal metaphors and
formally integrating a spectrum of human-AI collaboration models into
its structure \cite{arxiv2025human}. The introduction of AI does not merely add a new tool to
a designer\textquotesingle s toolkit; it fundamentally shifts the nature
of the design process itself, from a linear execution of steps to a
dynamic management of a complex workflow. This necessitates a paradigm
shift from process visualization to workflow
orchestration \cite{ibm2025orchestration}.

To this end, we propose a new framework, \textbf{The Orchestrated
Diamond}, which reframes the design process not as a static, sequential
map, but as a dynamic orchestration of human and machine intelligence.
This framework retains the familiar four phases of the Double Diamond
but reconceptualizes them as distinct modes of collaboration, each
requiring a specific and appropriate form of human-AI partnership.

The research will proceed as follows. Section 2 provides a critical
deconstruction of the Double Diamond, examining its history, principles,
and well-documented limitations, thereby establishing the clear need for
its evolution. Section 3 develops a formal taxonomy of human-AI
collaboration models, moving from the "AI Sandwich" metaphor to a
rigorous analysis of Human-in-the-Loop (HITL), AI-in-the-Loop (AITL),
and Mixed-Initiative systems to provide a solid theoretical foundation.
Section 4 details the proposed Orchestrated Diamond framework phase by
phase, specifying the nature of the human-AI collaboration at each
stage. Section 5 validates the framework by articulating its benefits,
acknowledging its inherent challenges, and drawing parallels to
empirical evidence and case studies where elements of this orchestrated
approach are already yielding significant value. Finally, Section 6
discusses the broader implications of this framework for design
practice, terminology, and future research in an era of increasingly
agentic AI.

\section{A Critical Re-examination
of the Double Diamond
Process}\label{section-2-a-critical-re-examination-of-the-double-diamond-process}

\subsection{Origins and Enduring
Principles}\label{origins-and-enduring-principles}

The Double Diamond design process model, while widely associated with
the British Design Council, has a rich intellectual lineage. The Council
popularized the model in 2004-2005 as a way to codify and visualize the
design processes used within the organization and make them accessible
to a broader audience.\cite{wikipedia2025double1} The model itself was an
adaptation of the divergence-convergence model proposed in 1996 by the
Hungarian-American linguist and systems scientist Béla H.
Bánáthy.\cite{wikipedia2025double1} Bánáthy\textquotesingle s work, in turn,
built upon decades of research into problem-solving and design theory
from luminaries such as John Dewey, who emphasized learning through
inquiry; Alex Osborn and Sidney Parnes, who proposed divergent and
convergent activities in their Creative Problem Solving model; and
Herbert Simon, who defined design as devising actions to change existing
situations into preferred ones \cite{medium2025double}.

The enduring power of the Double Diamond lies in its elegant simplicity.
It presents the design process through two conjoined diamonds, each
representing a cycle of divergent thinking (exploring an issue widely)
followed by convergent thinking (taking focused
action) \cite{wikipedia2025double2}. This structure is broken down into four
memorable, alliterative phases \cite{designcouncil2025history}:

\begin{enumerate}
\def\labelenumi{\arabic{enumi}.}
\item
  \textbf{Discover:} The first divergent phase, focused on understanding
  the issue through user research methods like interviews, surveys, and
  field studies, rather than making assumptions \cite{wikipedia2025double2}.
\item
  \textbf{Define:} The first convergent phase, where insights from
  discovery are synthesized to frame a clear, actionable problem
  definition or design brief \cite{wikipedia2025double2}.
\item
  \textbf{Develop:} The second divergent phase, focused on generating a
  wide range of potential solutions to the defined problem, often
  through co-design and seeking inspiration from diverse
  sources \cite{wikipedia2025double2}.
\item
  \textbf{Deliver:} The second convergent phase, where different
  solutions are tested at a small scale, refined, and prepared for
  launch \cite{wikipedia2025double2}.
\end{enumerate}

The primary objective of the Design Council was to create a framework
that was easy to memorize, share, and use to contextualize the work of
designers for clients and non-designers \cite{designcouncil2025history}. In this, it
has been remarkably successful, becoming an accepted part of the global
design language \cite{designcouncil2025history}.

\subsection{A Litany of Limitations in
Modern
Practice}\label{a-litany-of-limitations-in-modern-practice}

Despite its widespread adoption, the Double Diamond has faced persistent
and significant criticism, particularly as the complexity of design
challenges and the pace of development have increased. These critiques
reveal a model that, while conceptually elegant, often struggles to
align with the realities of modern product and service development.

A primary critique is the model\textquotesingle s \textbf{linearity and
abstraction}. The clean, sequential diagram belies the often messy,
chaotic, and iterative nature of real-world creative
work \cite{medium2025double}. Design is rarely a straight-lined process; it
involves false starts, backtracking, and multiple cycles of reframing as
new insights emerge.\cite{danramsden2025limitations} The model\textquotesingle s
high level of abstraction, while making it memorable, renders it less
useful for answering specific, practical questions about process. As one
critic notes, it often becomes a "black box masquerading as a glass
box," obscuring the actual art and craft of design behind a facade of
ordered simplification.\cite{danramsden2025limitations} This linearity is a holdover
from a more waterfall-style approach to projects, which is fundamentally
at odds with the agile and iterative methods prevalent in software and
digital product development today.

Perhaps the most damaging critique is the \textbf{feasibility and
reality gap}. The model is heavily centered on user needs, with little
to no formal consideration for technical feasibility, budget
constraints, or business viability until late in the
process \cite{reddit2025double}. This creates what practitioners describe as
a "massive disconnect between what users want, and what can
realistically be implemented" \cite{reddit2025double}. In many projects,
teams spend the first diamond deeply understanding users and the second
diamond brainstorming innovative solutions, only to discover during the
"Deliver" phase that "75\% of the ideas we came up with would need to be
custom built, and will therefore cost too much" \cite{reddit2025double}.
This often results in teams discarding their ambitious, user-centric
designs and reverting to standard, out-of-the-box UI and features,
rendering much of the initial discovery work moot \cite{reddit2025double}.
This flaw is particularly acute in corporate environments where
stakeholders often have a solution in mind and view extensive,
open-ended discovery as "unnecessary friction" or a "waste of
time".\cite{smartinterface2025double}

Furthermore, the original model suffers from a \textbf{lack of
measurement and post-launch feedback loops}. It was conceived as an
"idea funnel that transmutes questions into solutions," but without an
explicit mechanism for measuring the success of those solutions in the
real world \cite{medium2025double}. In practice, design work is often
considered complete once a prototype is tested and handed off to
development \cite{uxdesign2025why}. However, this ignores the crucial
design work that happens post-hand-off, as inevitable edge cases, error
states, and technical hurdles emerge during the build phase. Without a
formal process for continuous design involvement, this work is often
rushed or squeezed in, leading to a degradation in quality and strained
relationships between design and development teams \cite{uxdesign2025why}.

Finally, the model is often seen as \textbf{insufficient for the
complexity of modern systems design}. The challenges designers face
today with intricate services and systems have left the Double Diamond
"a bit short of breath" \cite{medium2025double}. Its simple structure does
not adequately account for the high levels of ambiguity and
non-linearity inherent in discontinuous innovation, nor does it
explicitly mention other crucial forms of design thinking, such as
abductive reasoning, or foundational concepts like
empathy \cite{researchgate2025design}. This has led even the Design Council itself
to explore successors, such as the Systemic Design Framework, that are
better equipped to handle these complexities \cite{medium2025double}.

The common thread running through all these critiques is a fundamental
temporal disconnect. The model\textquotesingle s structure rigidly
separates problem-finding (the first diamond) from solution-building
(the second diamond). This sequential, waterfall-like separation is the
root cause of its major flaws. If technical feasibility (a
solution-space constraint) were considered during the problem definition
phase, the feasibility gap would be mitigated. If feedback from the
development and delivery phases could easily and continuously loop back
to refine the problem definition, the lack of iterative feedback would
be solved. The core issue is not the four phases themselves, but their
strict segregation into two distinct, sequential stages.

\subsection{Reactive Evolution: The
Emergence of the Triple
Diamond}\label{reactive-evolution-the-emergence-of-the-triple-diamond}

In response to these well-documented failings, the design community has
proposed various extensions, most notably under the umbrella of the
"Triple Diamond".\cite{researchgate2025triple} It is not a single, unified model
but rather a collection of interpretations that attempt to patch the
perceived holes in the original framework.\cite{equalexperts2025innovation}

One common interpretation adds a third diamond focused on
\textbf{experimentation and implementation}. This version, proposed by
practitioners like Adam Gray, inserts a new diamond between "Develop"
and "Deliver" or expands "Deliver" into a full diamond of its
own.\cite{smartinterface2025double} The goal is to move beyond a simple tested
prototype to a live, released MVP. This third diamond encompasses the
detailed design of all screens, flows, and copy, followed by the build
and the subsequent gathering of qualitative and quantitative data from a
live release \cite{uxdesign2025why}. This directly addresses the critique
that the Double Diamond ends too early and lacks a robust post-launch
feedback mechanism.

A second interpretation places an additional diamond \emph{before} the
traditional two, focusing on \textbf{opportunity and
strategy}.\cite{equalexperts2025innovation} This version acknowledges that a design
process doesn\textquotesingle t begin with a given problem; it begins
with the strategic work of identifying which problem is worth solving in
the first place. This initial diamond would involve divergent thinking
to identify potential opportunities and convergent thinking to define a
clear strategy and business case before the "Discover" phase even
begins.

A third variation, described by Sophia Prater, inserts a central diamond
focused on \textbf{structuring insights} \cite{medium2025double}. This
diamond sits between "Define" and "Develop" and is dedicated to the
rigorous work of structuring research insights to properly inform the
design of complex systems, particularly around information architecture.

While these Triple Diamond models offer valuable additions by
formalizing stages that were implicit or missing in the original, they
are ultimately additive patches. They extend the linear chain of the
process---adding a stage before, after, or in the middle---but they do
not fundamentally alter its sequential nature. They do not create the
kind of dynamic, interwoven feedback loops that are necessary for truly
agile and responsive development in a technologically complex
environment. A more radical evolution is required, one that does not
simply add another diamond but weaves new capabilities throughout the
entire process, breaking down the rigid walls between the phases. This
is the opportunity that the structured integration of AI presents.

\section{A Taxonomy of Human-AI
Collaboration in Creative and Analytical
Workflows}\label{section-3-a-taxonomy-of-human-ai-collaboration-in-creative-and-analytical-workflows}

\subsection{Deconstructing the "AI
Sandwich": From Metaphor to
Model}\label{deconstructing-the-ai-sandwich-from-metaphor-to-model}

The "AI Sandwich" has emerged as a popular and intuitive metaphor for
describing a collaborative human-AI workflow \cite{nationaljurist2025ai}. Its
common interpretation involves a three-part structure: a human provides
the initial input, direction, or question (the top slice of bread); the
AI system performs a task, such as generating text, analyzing data, or
creating an image (the filling); and the human returns to critique,
refine, fact-check, and provide the final judgment (the bottom
slice) \cite{nationaljurist2025ai}. This simple concept has proven remarkably
versatile, finding application in diverse domains. In legal education,
Professor Dyane O\textquotesingle Leary uses it to teach law students
that they must be the "bread on both sides," responsible for both the
initial framing of a legal query and the critical analysis of the
AI\textquotesingle s output \cite{nationaljurist2025ai}. In creative contexts,
the human is described as the "cheese in the sandwich," making the
crucial decisions between AI-driven perception (research) and AI-driven
action (writing).\cite{medium2025symbiotic}

While powerful as a mental model, the "sandwich" metaphor lacks the
precision required for building robust, scalable systems. A more
formalized approach can be seen in the concept of a "Model Context
Protocol" (MCP).\cite{bilotta2025model} An MCP is a structured framework
for managing the information fed to an AI. It acts as a detailed "recipe
card" (the protocol) and a well-organized set of ingredients (the
context), ensuring that the AI\textquotesingle s output is consistent
and aligned with the user\textquotesingle s intent. The MCP framework
breaks down context into four key categories: System Context (what the
AI always knows), User Context (personalization), Conversation Context
(the ongoing interaction), and Task Context (the immediate
goal).\cite{bilotta2025model} This moves the discourse from a simple
metaphor to a structured model of interaction. By formally classifying
this pattern within established Human-Computer Interaction (HCI) theory,
we can see that the "AI Sandwich" is a practical, user-friendly
description of an AI-in-the-Loop (AITL) workflow. The human is in
control, initiating the process and making the final judgment, which
directly maps to the AITL definition where the human retains ultimate
decision-making authority \cite{arxiv2025human}.

\subsection{The Spectrum of Agency: HITL
vs.
AITL}\label{the-spectrum-of-agency-hitl-vs.-aitl}

To build a more sophisticated framework, it is essential to move beyond
a single interaction pattern and understand the full spectrum of
human-AI collaboration \cite{researchgate2025collaboration}. The most fundamental distinction in HCI
literature is based on the locus of control and decision-making
authority, primarily captured in the concepts of Human-in-the-Loop
(HITL) and AI-in-the-Loop (AITL) \cite{arxiv2025human}.

\textbf{Human-in-the-Loop (HITL)} systems are those in which the AI
system drives the decision-making process, but a human is integrated
into the loop to provide supervision, validation, or intervention for
edge cases \cite{arxiv2025human}. In this model, the AI is in control of
the operational flow, and human input is used to "guide" the model,
correct its errors, or handle tasks that fall below a certain confidence
threshold \cite{arxiv2025human}. This architecture is prevalent in
applications where accuracy and reliability are paramount, such as
medical imaging analysis, content moderation, and the labeling of
training data for supervised machine learning.\cite{googlecloud2025hitl} The
human acts as an oracle, a weak supervisor, or a final quality-control
gatekeeper \cite{arxiv2025human}.

\textbf{AI-in-the-Loop (AITL)} systems, conversely, invert this
relationship \cite{arxiv2025human}. In an AITL model, the human retains
full control and ultimate decision-making authority, while the AI acts
as an assistant or an augmentative tool to enhance the
human\textquotesingle s capabilities \cite{arxiv2025human}. The AI system
provides decision support, automates routine sub-tasks, or surfaces
insights for human interpretation, but it does not make the final
decision \cite{ibm2025ai}. This is the paradigm of AI as a smart
tool---a sophisticated calculator or a research assistant---that
enhances human performance without usurping human agency. As
established, the "AI Sandwich" is a classic AITL pattern.

The distinction between HITL and AITL is not merely academic; it is a
critical design choice that dictates the architecture, user interface,
and risk profile of a system \cite{arxiv2025human}. The decision hinges on
whether the primary goal is to automate a process with human oversight
(HITL) or to augment human judgment with AI assistance (AITL).

\subsection{Towards Partnership:
Mixed-Initiative and Collaborative
Intelligence}\label{towards-partnership-mixed-initiative-and-collaborative-intelligence}

The HITL and AITL models, while useful, primarily describe a
master-servant dynamic. More advanced forms of collaboration aim for a
true partnership, where the roles are more fluid and synergistic.

\textbf{Mixed-Initiative (MI)} systems are defined by the ability of
both the human and the AI to proactively contribute to a shared goal,
with the conversational or creative "initiative" shifting between
them.\cite{aimagazine2025mixed} In an MI system, the AI is not merely
responding to explicit commands; it can take the lead, make unsolicited
suggestions, propose alternative paths, and actively shape the direction
of the work.\cite{goldsmiths2025mixed} This creates a dynamic, peer-to-peer
interaction that is essential for co-creative tasks like brainstorming,
game level design, or collaborative writing, where the goal is to
generate novel artifacts through a real-time, improvisational
partnership.\cite{researchgate2025boosting} The system and the human become
collaborators who iteratively build upon each other\textquotesingle s
contributions.\cite{researchgate2025implications}

\textbf{Collaborative Intelligence (CI)} and \textbf{Human-AI Teaming
(HAIT)} represent a socio-technical perspective on this partnership,
viewing the AI not just as a tool or a system, but as a genuine team
member.\cite{pmc2025defining} CI is defined by the synergistic combination
of human and AI capabilities to achieve outcomes that would be
unattainable by either agent alone.\cite{tandfonline2025collaborative} This approach
focuses on augmenting human intelligence, freeing people from
automatable and repetitive tasks so they can focus on their unique
strengths: creativity, strategic thinking, empathy, and ethical
judgment.\cite{smythos2025exploring} Successful HAIT depends on factors that
are familiar from human teams: shared goals, interdependence, clear
communication, evolving shared mental models, and, crucially, mutual
trust.\cite{pmc2025defining}

These different models are not mutually exclusive categories but rather
points along a \textbf{spectrum of agency}. This spectrum progresses
from low AI agency (AITL, where the AI is a responsive tool), to
moderate AI agency (HITL, where the AI can execute processes but is
gated by human approval), to peer-level agency (Mixed-Initiative, where
the AI is a proactive partner), and ultimately towards fully agentic AI
(where the AI can set and pursue its own goals with minimal
oversight).\cite{nice2025forrester} A sophisticated design framework should
not be locked into a single mode of interaction. Instead, it should
strategically select the appropriate level of AI agency for the specific
task at hand within the broader workflow. This dynamic selection is the
core principle that enables the transition from a simple design process
to an orchestrated one.

\textbf{Table 1: A Comparative Analysis of Human-AI Interaction Models}

\begin{longtable}[]{@{}
  >{\raggedright\arraybackslash}p{(\columnwidth - 12\tabcolsep) * \real{0.1429}}
  >{\raggedright\arraybackslash}p{(\columnwidth - 12\tabcolsep) * \real{0.1429}}
  >{\raggedright\arraybackslash}p{(\columnwidth - 12\tabcolsep) * \real{0.1429}}
  >{\raggedright\arraybackslash}p{(\columnwidth - 12\tabcolsep) * \real{0.1429}}
  >{\raggedright\arraybackslash}p{(\columnwidth - 12\tabcolsep) * \real{0.1429}}
  >{\raggedright\arraybackslash}p{(\columnwidth - 12\tabcolsep) * \real{0.1429}}
  >{\raggedright\arraybackslash}p{(\columnwidth - 12\tabcolsep) * \real{0.1429}}@{}}
\toprule\noalign{}
\begin{minipage}[b]{\linewidth}\raggedright
Model
\end{minipage} & \begin{minipage}[b]{\linewidth}\raggedright
Primary Locus of Control
\end{minipage} & \begin{minipage}[b]{\linewidth}\raggedright
AI\textquotesingle s Role
\end{minipage} & \begin{minipage}[b]{\linewidth}\raggedright
Human\textquotesingle s Role
\end{minipage} & \begin{minipage}[b]{\linewidth}\raggedright
Core Analogy
\end{minipage} & \begin{minipage}[b]{\linewidth}\raggedright
Primary Goal
\end{minipage} & \begin{minipage}[b]{\linewidth}\raggedright
Supporting Sources
\end{minipage} \\
\begin{minipage}[b]{\linewidth}\raggedright
\textbf{AI-in-the-Loop (AITL)}
\end{minipage} & \begin{minipage}[b]{\linewidth}\raggedright
Human
\end{minipage} & \begin{minipage}[b]{\linewidth}\raggedright
Assistant / Tool / Augmentation Layer
\end{minipage} & \begin{minipage}[b]{\linewidth}\raggedright
Decision-Maker / Orchestrator / User
\end{minipage} & \begin{minipage}[b]{\linewidth}\raggedright
AI as a smart tool (e.g., spell-checker, research assistant)
\end{minipage} & \begin{minipage}[b]{\linewidth}\raggedright
Augment Human Performance
\end{minipage} & \begin{minipage}[b]{\linewidth}\raggedright
\cite{arxiv2025human}
\end{minipage} \\
\begin{minipage}[b]{\linewidth}\raggedright
\textbf{Human-in-the-Loop (HITL)}
\end{minipage} & \begin{minipage}[b]{\linewidth}\raggedright
AI
\end{minipage} & \begin{minipage}[b]{\linewidth}\raggedright
Driver / Executor / Autonomous Process
\end{minipage} & \begin{minipage}[b]{\linewidth}\raggedright
Supervisor / Validator / Exception Handler
\end{minipage} & \begin{minipage}[b]{\linewidth}\raggedright
AI as a self-driving car with a human monitor
\end{minipage} & \begin{minipage}[b]{\linewidth}\raggedright
Ensure Accuracy \& Reliability
\end{minipage} & \begin{minipage}[b]{\linewidth}\raggedright
\cite{arxiv2025human}
\end{minipage} \\
\begin{minipage}[b]{\linewidth}\raggedright
\textbf{Mixed-Initiative (MI)}
\end{minipage} & \begin{minipage}[b]{\linewidth}\raggedright
Shifting / Shared
\end{minipage} & \begin{minipage}[b]{\linewidth}\raggedright
Partner / Collaborator / Co-Creator
\end{minipage} & \begin{minipage}[b]{\linewidth}\raggedright
Co-Creator / Peer / Creative Director
\end{minipage} & \begin{minipage}[b]{\linewidth}\raggedright
AI as a brainstorming partner or duet musician
\end{minipage} & \begin{minipage}[b]{\linewidth}\raggedright
Co-Create Novelty \& Explore Possibilities
\end{minipage} & \begin{minipage}[b]{\linewidth}\raggedright
\cite{researchgate2025boosting}
\end{minipage} \\
\begin{minipage}[b]{\linewidth}\raggedright
\textbf{Collaborative Intelligence (CI) / Human-AI Teaming (HAIT)}
\end{minipage} & \begin{minipage}[b]{\linewidth}\raggedright
Shared
\end{minipage} & \begin{minipage}[b]{\linewidth}\raggedright
Synergistic Teammate / Cognitive Enhancer
\end{minipage} & \begin{minipage}[b]{\linewidth}\raggedright
Strategist / Ethical Guide / Team Leader
\end{minipage} & \begin{minipage}[b]{\linewidth}\raggedright
AI as a full team member with specialized skills
\end{minipage} & \begin{minipage}[b]{\linewidth}\raggedright
Achieve Superior Synergistic Outcomes
\end{minipage} & \begin{minipage}[b]{\linewidth}\raggedright
\cite{pmc2025defining}
\end{minipage} \\
\midrule\noalign{}
\endhead
\bottomrule\noalign{}
\endlastfoot
\end{longtable}

\section{The Orchestrated Diamond:
A Proposed Framework for AI-Integrated
Design}\label{section-4-the-orchestrated-diamond-a-proposed-framework-for-ai-integrated-design}

\subsection{Framework Overview: From
Sequential Phases to Orchestrated
Modes}\label{framework-overview-from-sequential-phases-to-orchestrated-modes}

The proposed framework, The Orchestrated Diamond, retains the four
familiar phases of the Double Diamond---Discover, Define, Develop, and
Deliver---for their established conceptual clarity and mnemonic
value \cite{designcouncil2025history}. The fundamental innovation of this framework
lies in the reconceptualization of these phases. They are no longer
treated as rigid, linear steps in a process but as distinct

\textbf{modes of collaboration}, each requiring a different
configuration of human-AI interaction. This approach draws directly from
the taxonomy of collaboration models established in Section 3. The role
of the designer is elevated from that of a process-follower to that of
an \textbf{orchestrator}, who strategically manages the workflow and
facilitates the transitions between these collaborative modes. This
dynamic structure directly addresses the core critique of the Double
Diamond\textquotesingle s restrictive linearity and its failure to
represent the iterative nature of modern design
work \cite{medium2025double}.

\subsection{Phase 1 - Discover:
AI-Assisted Inquiry (AITL
Mode)}\label{phase-1---discover-ai-assisted-inquiry-aitl-mode}

The first phase of the framework is dedicated to the divergent
exploration of the problem space. The goal is to cast a wide net,
gathering as much information as possible to understand the context,
user needs, and market landscape.

\begin{itemize}
\item
  \textbf{Collaboration Model:} This phase operates in an
  \textbf{AI-in-the-Loop (AITL)} mode. The human researcher or designer
  maintains full strategic control over the inquiry, defining the
  research questions, setting the scope, and critically evaluating the
  information that is gathered. The AI functions as a powerful and
  tireless research assistant, dramatically accelerating the speed and
  expanding the breadth of the discovery process.
\item
  \textbf{Human Role:} The human acts as a \textbf{Strategist,
  Researcher, and Prompter}. They are responsible for designing the
  research plan, formulating precise queries for the AI systems, and
  applying their domain expertise to interpret the outputs.
\item
  \textbf{AI Role:} The AI\textquotesingle s role is that of a
  \textbf{Research Accelerator}. It automates and augments
  data-intensive research tasks that are traditionally time-consuming
  and labor-intensive. Specific AI-powered tasks include:

  \begin{itemize}
  \item
    \textbf{Large-Scale Data Analysis:} AI algorithms can analyze vast,
    unstructured datasets, such as thousands of customer support
    tickets, online reviews, or social media comments, to perform
    sentiment analysis and identify recurring themes or pain
    points.\cite{cogleus2025rise}
  \item
    \textbf{User Research Augmentation:} Generative AI can assist in
    creating initial drafts of interview scripts or survey
    questions.\cite{cuny2025combining} After interviews are conducted,
    AI-powered tools can provide automated transcriptions and summaries,
    freeing the researcher to focus on the interaction itself and
    accelerating the initial analysis.\cite{ibm2025workflow}
  \item
    \textbf{Market and Competitive Analysis:} AI can rapidly scan and
    synthesize information about competitor products, market trends, and
    relevant academic or industry literature, providing a comprehensive
    landscape analysis in a fraction of the time required for manual
    research.\cite{maze2025double}
  \end{itemize}
\item
  \textbf{Connection to DD Critique:} This mode directly enhances the
  traditional "Discover" phase. By leveraging AI, a design team can
  conduct a far wider and deeper divergent search than is manually
  feasible within typical project constraints, leading to a richer and
  more evidence-based understanding of the problem space from the
  outset.
\end{itemize}

\subsection{Phase 2 - Define: Human-Led
Synthesis (AITL/Human-Control
Mode)}\label{phase-2---define-human-led-synthesis-aitlhuman-control-mode}

The second phase is a convergent activity focused on making sense of the
vast information gathered during discovery and distilling it into a
clear, concise, and actionable problem statement.

\begin{itemize}
\item
  \textbf{Collaboration Model:} This phase operates primarily in an AITL
  mode but with an explicit emphasis on \textbf{Human-Led Synthesis}.
  While AI can assist in organizing data, the ultimate act of defining
  the problem---of framing the challenge based on strategic goals,
  ethical considerations, and empathetic understanding---is a uniquely
  human responsibility that cannot be delegated.
\item
  \textbf{Human Role:} The human is the \textbf{Synthesizer,
  Sense-Maker, and Decision-Maker}. They use their critical thinking,
  domain expertise, and strategic insight to interpret the patterns
  surfaced by the AI, identify the core user need or opportunity, and
  frame the definitive problem statement that will guide the rest of the
  project.\cite{danramsden2025limitations}
\item
  \textbf{AI Role:} The AI serves as a \textbf{Pattern Recognition
  Tool}. It helps to structure and process the large volume of raw data
  from the Discover phase, making it more manageable for human analysis.
  Specific AI-powered tasks include:

  \begin{itemize}
  \item
    \textbf{AI-Assisted Affinity Mapping:} Instead of manually sorting
    hundreds of digital sticky notes, AI can process raw research data
    (e.g., interview excerpts, survey responses) and suggest thematic
    clusters or affinity groups, significantly accelerating the
    synthesis process.\cite{maze2025double}
  \item
    \textbf{Insight Identification:} AI can perform statistical analysis
    to identify correlations or anomalies in the data that might not be
    immediately apparent to a human analyst, pointing to potential areas
    of interest.\cite{cogleus2025rise}
  \item
    \textbf{Problem Statement Ideation:} Once the human has identified
    the core insight, generative AI can be used as a tool to brainstorm
    multiple phrasings of the problem statement or to generate a range
    of "How Might We" (HMW) questions to explore different angles of the
    challenge.\cite{cuny2025combining}
  \end{itemize}
\item
  \textbf{Connection to DD Critique:} This mode directly addresses the
  critique that the Double Diamond is too abstract and can lead to a
  "disordered mess".\cite{danramsden2025limitations} By using AI to bring structure
  to the chaos of raw research data, the framework allows the human
  designer to focus on the higher-order cognitive task of synthesis and
  definition, resulting in a more concrete, evidence-backed problem
  statement.
\end{itemize}

\subsection{Phase 3 - Develop:
Mixed-Initiative Co-Creation (MI
Mode)}\label{phase-3---develop-mixed-initiative-co-creation-mi-mode}

The third phase marks the second point of divergence, where the team
explores a wide array of potential solutions to the clearly defined
problem.

\begin{itemize}
\item
  \textbf{Collaboration Model:} This phase shifts into a
  \textbf{Mixed-Initiative (MI)} mode. This is a true human-AI
  partnership where both agents can proactively contribute ideas and
  build upon each other\textquotesingle s work. The creative initiative
  flows back and forth between the human designer and the AI system.
\item
  \textbf{Human Role:} The human acts as a \textbf{Creative Director,
  Prompter, Curator, and Collaborator}. They set the creative direction,
  define the constraints for the AI, provide initial sparks of
  inspiration, and then curate, combine, and refine the multitude of
  options generated by the AI partner.\cite{researchgate2025implications}
\item
  \textbf{AI Role:} The AI functions as a \textbf{Creative Partner and
  Ideation Engine}. Its role is not just to execute commands but to
  actively participate in the creative process. Specific AI-powered
  tasks include:

  \begin{itemize}
  \item
    \textbf{Generative Design and Ideation:} AI systems can generate a
    vast number of design concepts based on a set of constraints (e.g.,
    materials, cost, performance metrics, aesthetic style) provided by
    the human designer. This applies to UI layouts, product forms,
    architectural plans, and more.\cite{invozone2025collaborative} This capability
    helps teams break free from cognitive biases and explore a much
    broader solution space.\cite{ideou2025intersection}
  \item
    \textbf{Augmented Brainstorming:} The AI can serve as an
    inexhaustible brainstorming partner, suggesting unconventional
    solutions, providing visual inspiration, or developing variations on
    a human-generated idea.\cite{smythos2025exploring}
  \item
    \textbf{Rapid Visualization:} Generative AI tools like Midjourney or
    Stable Diffusion can instantly create high-fidelity visualizations
    of abstract concepts, facilitating clearer communication within the
    team and allowing for faster evaluation and iteration of
    ideas.\cite{researchgate2025thinking}
  \end{itemize}
\item
  \textbf{Connection to DD Critique:} This mode provides a powerful
  solution to the DD\textquotesingle s critical feasibility
  gap \cite{reddit2025double}. By incorporating technical, financial, and
  material constraints directly into the generative design prompts,
  feasibility becomes an integral part of the ideation process, not an
  afterthought. This ensures that the divergent exploration of solutions
  is grounded in real-world viability from the start.
\end{itemize}

\subsection{Phase 4 - Deliver:
AI-Augmented Validation (HITL
Mode)}\label{phase-4---deliver-ai-augmented-validation-hitl-mode}

The final phase is one of convergence, where the most promising
solutions from the Develop phase are prototyped, tested, refined, and
prepared for implementation.

\begin{itemize}
\item
  \textbf{Collaboration Model:} This phase primarily utilizes a
  \textbf{Human-in-the-Loop (HITL)} model. AI systems are given the
  autonomy to execute complex and repetitive tasks, such as generating
  code for prototypes or analyzing user testing data. However, the human
  designer must remain "in the loop" to validate the outputs, interpret
  the results, provide quality control, and make the final decisions on
  the solution\textquotesingle s viability.
\item
  \textbf{Human Role:} The human is the \textbf{Validator, Ethical
  Overseer, and Quality Control Manager}. They are responsible for
  reviewing AI-generated prototypes for accuracy and usability,
  interpreting the nuances of user feedback that an AI might miss,
  ensuring the final product is ethical and robust, and making the
  ultimate go/no-go decision \cite{uxdesign2025why}.
\item
  \textbf{AI Role:} The AI\textquotesingle s role is that of a
  \textbf{Prototyping and Testing Engine}. It dramatically accelerates
  the iterative cycles of building and testing. Specific AI-powered
  tasks include:

  \begin{itemize}
  \item
    \textbf{Rapid Prototyping and Code Generation:} AI tools can
    transform static design mockups into interactive prototypes or even
    generate front-end code, significantly reducing development time for
    testable artifacts.\cite{logrocket2025designers}
  \item
    \textbf{Automated and Predictive Testing:} AI can automate various
    forms of testing, such as simulating diverse user scenarios,
    checking for accessibility compliance, or analyzing A/B test
    results.\cite{ideou2025intersection} Predictive analytics can be used on
    prototype usage data to forecast user satisfaction or identify
    potential friction points before a full-scale
    launch.\cite{logrocket2025designers}
  \item
    \textbf{Feedback Analysis:} AI can process large volumes of user
    feedback from testing sessions, summarizing key themes and
    sentiments to help designers quickly identify areas for
    improvement.\cite{ipm2025impact}
  \end{itemize}
\item
  \textbf{Connection to DD Critique:} This mode directly addresses the
  Double Diamond\textquotesingle s lack of explicit and efficient
  feedback loops \cite{medium2025double}. The sheer speed of AI-augmented
  prototyping and validation enables teams to conduct many more, and
  much faster, iterative cycles than is possible with traditional
  methods, leading to a more refined and robust final product.
\end{itemize}

\textbf{Table 2: Mapping AI Roles and Tools to the Orchestrated Diamond
Framework}

\begin{longtable}[]{@{}
  >{\raggedright\arraybackslash}p{(\columnwidth - 10\tabcolsep) * \real{0.1667}}
  >{\raggedright\arraybackslash}p{(\columnwidth - 10\tabcolsep) * \real{0.1667}}
  >{\raggedright\arraybackslash}p{(\columnwidth - 10\tabcolsep) * \real{0.1667}}
  >{\raggedright\arraybackslash}p{(\columnwidth - 10\tabcolsep) * \real{0.1667}}
  >{\raggedright\arraybackslash}p{(\columnwidth - 10\tabcolsep) * \real{0.1667}}
  >{\raggedright\arraybackslash}p{(\columnwidth - 10\tabcolsep) * \real{0.1667}}@{}}
\toprule\noalign{}
\begin{minipage}[b]{\linewidth}\raggedright
Phase
\end{minipage} & \begin{minipage}[b]{\linewidth}\raggedright
Core Activity
\end{minipage} & \begin{minipage}[b]{\linewidth}\raggedright
AI Collaboration Model
\end{minipage} & \begin{minipage}[b]{\linewidth}\raggedright
Human\textquotesingle s Primary Role
\end{minipage} & \begin{minipage}[b]{\linewidth}\raggedright
AI\textquotesingle s Primary Role
\end{minipage} & \begin{minipage}[b]{\linewidth}\raggedright
Example AI-Powered Tasks \& Tools
\end{minipage} \\
\begin{minipage}[b]{\linewidth}\raggedright
\textbf{Discover}
\end{minipage} & \begin{minipage}[b]{\linewidth}\raggedright
Divergent Research
\end{minipage} & \begin{minipage}[b]{\linewidth}\raggedright
AI-in-the-Loop (AITL)
\end{minipage} & \begin{minipage}[b]{\linewidth}\raggedright
Research Strategist
\end{minipage} & \begin{minipage}[b]{\linewidth}\raggedright
Research Accelerator
\end{minipage} & \begin{minipage}[b]{\linewidth}\raggedright
- Market trend analysis with predictive analytics platforms. - Sentiment
analysis of user reviews using NLP tools. - Automated transcription and
summarization of user interviews (e.g., Otter.ai).
\end{minipage} \\
\begin{minipage}[b]{\linewidth}\raggedright
\textbf{Define}
\end{minipage} & \begin{minipage}[b]{\linewidth}\raggedright
Convergent Synthesis
\end{minipage} & \begin{minipage}[b]{\linewidth}\raggedright
Human-Led Synthesis (AITL)
\end{minipage} & \begin{minipage}[b]{\linewidth}\raggedright
Sense-Maker \& Decider
\end{minipage} & \begin{minipage}[b]{\linewidth}\raggedright
Pattern Recognition Tool
\end{minipage} & \begin{minipage}[b]{\linewidth}\raggedright
- AI-assisted affinity clustering of research data. - Generating "How
Might We" questions from insights (e.g., ChatGPT). - Identifying data
correlations with business intelligence tools.
\end{minipage} \\
\begin{minipage}[b]{\linewidth}\raggedright
\textbf{Develop}
\end{minipage} & \begin{minipage}[b]{\linewidth}\raggedright
Divergent Ideation
\end{minipage} & \begin{minipage}[b]{\linewidth}\raggedright
Mixed-Initiative (MI)
\end{minipage} & \begin{minipage}[b]{\linewidth}\raggedright
Creative Director \& Curator
\end{minipage} & \begin{minipage}[b]{\linewidth}\raggedright
Creative Partner
\end{minipage} & \begin{minipage}[b]{\linewidth}\raggedright
- Generative design for physical products or architecture (e.g.,
Autodesk Forma). - Rapid concept visualization and mood boarding (e.g.,
Midjourney, Stable Diffusion). - Brainstorming unconventional solutions
with a large language model.
\end{minipage} \\
\begin{minipage}[b]{\linewidth}\raggedright
\textbf{Deliver}
\end{minipage} & \begin{minipage}[b]{\linewidth}\raggedright
Convergent Validation
\end{minipage} & \begin{minipage}[b]{\linewidth}\raggedright
Human-in-the-Loop (HITL)
\end{minipage} & \begin{minipage}[b]{\linewidth}\raggedright
Validator \& Ethical Overseer
\end{minipage} & \begin{minipage}[b]{\linewidth}\raggedright
Prototyping \& Testing Engine
\end{minipage} & \begin{minipage}[b]{\linewidth}\raggedright
- Transforming mockups into interactive prototypes (e.g., Uizard, Framer
AI). - Automated accessibility and usability testing. - Analyzing A/B
test results and predicting user behavior.
\end{minipage} \\
\midrule\noalign{}
\endhead
\bottomrule\noalign{}
\endlastfoot
\end{longtable}

\section{Framework Validation:
Benefits, Challenges, and Empirical
Parallels}\label{section-5-framework-validation-benefits-challenges-and-empirical-parallels}

\subsection{Articulated Benefits:
Addressing the Double Diamond\textquotesingle s
Deficiencies}\label{articulated-benefits-addressing-the-double-diamonds-deficiencies}

The proposed Orchestrated Diamond framework is not merely a theoretical
exercise; it is designed to directly address the most pressing and
well-documented deficiencies of the traditional Double Diamond model. By
structuring the design process as a series of distinct human-AI
collaborative modes, it offers tangible benefits.

First, the framework \textbf{overcomes the problem of rigid linearity}.
The original DD\textquotesingle s sequential nature is a poor fit for
agile environments \cite{medium2025double}. The Orchestrated Diamond is
inherently iterative. The speed and low cost of AI-augmented prototyping
and testing in the "Deliver" phase make it feasible to execute rapid
feedback loops. An insight from testing can immediately inform a new
round of ideation in the "Develop" phase or even trigger a reframing of
the problem in the "Define" phase, breaking down the artificial walls
between the diamonds.

Second, it systematically \textbf{bridges the feasibility gap}. A major
failure of the DD is its tendency to produce innovative but ultimately
unfeasible solutions by deferring technical and business constraints
until the end of the process \cite{reddit2025double}. The Orchestrated
Diamond integrates feasibility from the outset. In the "Develop" phase,
the Mixed-Initiative co-creation process can be constrained by
parameters such as cost, materials, or manufacturing techniques,
ensuring that the generated ideas are grounded in
reality.\cite{researchgate2025impact} Furthermore, the AI-Augmented Validation in
the "Deliver" phase allows for low-cost, high-fidelity simulation and
testing before significant engineering resources are committed,
de-risking the entire development lifecycle.\cite{logrocket2025designers}

Third, the framework \textbf{integrates measurement and validation as a
core component}. The original DD was criticized for lacking explicit
mechanisms for measuring success or iterating
post-launch \cite{medium2025double}. The Orchestrated
Diamond\textquotesingle s "Deliver" phase is built entirely around the
principles of testing, data analysis, and validation. This embeds a
data-driven, evidence-based culture into the design process, ensuring
that decisions are based on performance metrics rather than solely on
intuition. This structure provides the explicit feedback loops that the
original model lacks.\cite{ideou2025intersection}

\subsection{Inherent Challenges and
Mitigation
Strategies}\label{inherent-challenges-and-mitigation-strategies}

Adopting the Orchestrated Diamond framework is not without its own set
of challenges. The integration of AI introduces new complexities and
risks that must be proactively managed. However, the very structure of
the framework provides a means of mitigation.

A primary concern is \textbf{algorithmic bias and ethical integrity}. AI
models are trained on historical data and can inadvertently learn and
amplify existing societal biases related to race, gender, or other
protected characteristics.\cite{smythos2025exploring} An AI tool used in the
"Discover" phase might surface biased insights, or a generative tool in
the "Develop" phase might produce stereotypical imagery. The framework
mitigates this risk by formally defining the human\textquotesingle s
role as an

\textbf{Ethical Overseer}, particularly in the Human-Led Synthesis
("Define") and HITL Validation ("Deliver") phases. These phases act as
critical checkpoints where human judgment is explicitly required to
scrutinize AI outputs for fairness, transparency, and accountability, in
line with established ethical AI principles.\cite{unesco2025ethics}

Another significant challenge is \textbf{data quality and privacy}. The
efficacy of the entire framework is contingent on access to
high-quality, clean, and relevant data, which remains a major
organizational hurdle for many enterprises.\cite{researchgate2025design}
Furthermore, the use of user data raises significant privacy concerns
that require robust data governance and compliance with regulations like
GDPR.\cite{ideou2025intersection} The framework addresses this by making data
strategy an explicit consideration. The "Define" phase, for instance,
must include an assessment of data availability and quality as a key
constraint on the problem definition.

Finally, there is the risk of \textbf{deskilling and over-reliance} on
AI. Practitioners and academics alike express concern that an
over-reliance on AI tools could lead to the atrophy of fundamental
design skills and a decline in critical thinking.\cite{ipm2025impact}
The Orchestrated Diamond counteracts this by framing AI as a
collaborator that elevates, rather than replaces, human expertise. The
framework assigns distinct and cognitively demanding roles to the human
at each stage---Strategist, Synthesizer, Creative Director, and
Validator. These roles require a higher level of strategic thinking and
critical judgment, shifting the designer\textquotesingle s value away
from mere execution and towards orchestration and curation. This
purposeful design of the human\textquotesingle s role is intended to
foster a culture of "collaborative intelligence" where the human-AI team
achieves more than either could alone.\cite{tandfonline2025collaborative}

While the framework introduces new risks, its structure also serves as a
risk mitigation tool. By mandating specific points of human control,
synthesis, and validation, it prevents the "black box" problem of
blindly trusting an AI from start to finish. It builds in the necessary
checkpoints for human judgment, making the process itself a protocol for
safety and ethical diligence.

\subsection{Empirical Parallels and Case
Studies}\label{empirical-parallels-and-case-studies}

The principles underpinning the Orchestrated Diamond are not merely
speculative; they are mirrored in the successful application of AI
across various industries. These real-world case studies provide
empirical validation for the framework\textquotesingle s phase-specific
collaboration models.

In \textbf{Product Development}, companies like BMW and Mercedes-Benz
exemplify the "Develop" and "Deliver" phases. They use AI-driven
generative design to explore thousands of optimized vehicle component
designs based on performance and material constraints, a clear example
of Mixed-Initiative Co-Creation.\cite{mindinventory2025role} They then use
AI-powered computer vision for automated quality assurance on the
assembly line and predictive maintenance in deployed vehicles, which
mirrors the AI-Augmented Validation of the "Deliver"
phase.\cite{mindinventory2025role}

In \textbf{Service and Experience Design}, streaming giants like Netflix
and Spotify demonstrate the full cycle. They use sophisticated AI to
analyze massive datasets of user behavior to discover viewing patterns
and define opportunities for new content or features
(Discover/Define).\cite{mindinventory2025role} They then use AI to generate
personalized content, such as custom thumbnails for movies or
algorithmically curated playlists, which is a form of co-creating the
user experience (Develop).\cite{logrocket2025designers} This shows a direct link
between the type of human-AI collaboration employed and the specific
business value generated---in this case, hyper-personalization and user
engagement \cite{ibm2025ai}.

In \textbf{Sustainable Architecture}, the framework\textquotesingle s
principles are used to tackle complex, multi-variable problems. AI tools
are used in the "Develop" phase to run simulations that generate
building designs optimized for energy efficiency, daylighting, and
sustainable material use, all within defined cost
constraints.\cite{aecassociates2025transforming} This is a powerful example of using
Mixed-Initiative Co-Creation to solve the DD\textquotesingle s
feasibility gap. AI-driven monitoring systems are then used in completed
buildings for predictive maintenance and real-time optimization of HVAC
and lighting systems, an application of the "Deliver"
phase\textquotesingle s validation and feedback
loop.\cite{aecassociates2025transforming}

Finally, in \textbf{Educational Content Development}, the entire
Orchestrated Diamond is visible. AI is used to analyze student
performance data to discover learning gaps and personalize learning
journeys (Discover/Define).\cite{vktr2025ai} It is then used to
generate draft content, quizzes, and instructional videos
(Develop).\cite{vktr2025ai} Finally, AI automates the assessment and
grading process, providing feedback to both students and educators,
which closes the loop in the "Deliver" phase.\cite{frontiers2025developing}

These cases demonstrate that a causal link exists between the
collaboration model and the value created. The Orchestrated Diamond is
the first framework to make this link explicit, providing a strategic
tool for design teams to select the right collaborative approach to
achieve a specific goal, whether it be efficiency, innovation, or
quality.

\section{Discussion: Evolving
Terminology, Practice, and Future
Trajectories}\label{section-6-discussion-evolving-terminology-practice-and-future-trajectories}

\subsection{The Lexicon of Collaboration:
Beyond
"Augmented"}\label{the-lexicon-of-collaboration-beyond-augmented}

The terminology used to describe new paradigms is critical, as it shapes
understanding and adoption. The initial working title, "AI-Augmented
Double Diamond" (AADD), while descriptive, is ultimately insufficient.
The term "augmented" implies that AI is solely a subordinate tool that
enhances an existing human-led process. This accurately describes the
AITL mode of the "Discover" and "Define" phases, but it fails to capture
the peer-level partnership of the Mixed-Initiative "Develop" phase or
the AI-driven execution in the HITL "Deliver" phase. The relationship is
more complex and dynamic than simple augmentation.

A more precise and powerful term is \textbf{"The Orchestrated Diamond."}
The concept of "orchestration" better reflects the
designer\textquotesingle s new, elevated role.\cite{uxdesign2025why} An
orchestrator is not just a performer but a conductor who manages a
complex system of agents---both human and artificial---and skillfully
guides their interaction. They are responsible for selecting the correct
collaborative mode for each phase of the work, managing the flow of data
and insights between them, and ensuring that all parts work in harmony
to achieve a unified goal.\cite{uxdesign2025why} This term correctly
positions the designer\textquotesingle s primary contribution as one of
strategic management and creative direction, a role that is more
critical than ever in an AI-infused environment. While alternatives like
"Mixed-Initiative Diamond" or "Collaborative Diamond" capture aspects of
the framework, "Orchestrated" uniquely conveys the active, managerial,
and dynamic nature of the entire process.

\subsection{The Evolving Designer: From
Creator to
Orchestrator}\label{the-evolving-designer-from-creator-to-orchestrator}

The integration of AI as a collaborative partner precipitates a profound
evolution in the role and skillset of the designer. The traditional
emphasis on solitary creation and craft is shifting towards a new set of
competencies centered on collaboration, strategy, and management. The
designer of the future is less of a hands-on creator and more of a
\textbf{strategic orchestrator} of a human-AI team.

This role demands a new set of core competencies:

\begin{itemize}
\item
  \textbf{AI Literacy and Prompt Engineering:} A fundamental
  understanding of how different AI models work, their strengths, and
  their limitations is now essential. The ability to craft effective
  prompts that clearly communicate intent, context, and constraints to
  an AI is becoming a critical design skill in
  itself.\cite{researchgate2025thinking}
\item
  \textbf{Systems Thinking:} The modern design process is no longer a
  linear path but a complex, interconnected workflow. Designers must be
  able to think in terms of systems, understanding how to chain together
  different AI tools, data pipelines, and human checkpoints to create a
  seamless and efficient end-to-end process.\cite{uxdesign2025why}
\item
  \textbf{Ethical Guardianship and Critical Judgment:} As AI takes on
  more of the generative work, the human\textquotesingle s role as an
  ethical guardian becomes paramount. Designers must be adept at
  identifying and mitigating algorithmic bias, ensuring data privacy,
  and making the final call on whether a solution is not just effective
  but also fair, transparent, and responsible.\cite{forrester2025design} This
  includes the crucial skill of curation---the ability to critically
  evaluate a sea of AI-generated options and apply human taste,
  judgment, and strategic insight to select the most promising path
  forward.
\item
  \textbf{Human-AI Teaming Skills:} Designers must learn how to
  effectively collaborate with non-human agents. This includes building
  trust in AI systems, understanding how to delegate tasks effectively,
  and fostering a shared mental model with their AI
  partners.\cite{pmc2025defining}
\end{itemize}

This evolution of the designer\textquotesingle s role is not unique; it
serves as a powerful template for the future of knowledge work in
general. Professionals in law, finance, marketing, and science will all
need to transition from being individual contributors to becoming
orchestrators of human-AI teams. The Orchestrated Diamond, therefore,
can be seen not just as a framework for design, but as a case study in
the future of collaborative, intelligent work.

\subsection{Future Trajectories: Agentic
AI and the "No-UI"
Paradigm}\label{future-trajectories-agentic-ai-and-the-no-ui-paradigm}

Looking forward, the Orchestrated Diamond framework is positioned to
evolve with the next wave of AI technology. The current landscape is
dominated by tools that perform specific tasks. The future points
towards the rise of \textbf{agentic AI}---integrated systems that can
autonomously understand high-level goals, break them down into
sub-tasks, select their own tools, and execute complex, multi-step plans
with minimal human intervention.\cite{nice2025forrester}

In such a future, the Orchestrated Diamond might evolve significantly.
Instead of a human designer orchestrating a collection of discrete AI
tools (a summarizer, an image generator, a testing tool), they might
collaborate with a single, powerful AI agent. This agent could fluidly
switch its own internal modes, acting as a research assistant, a
creative partner, and a testing engine as needed to accomplish the
overarching goal set by the human.\cite{openai2025chatgpt} The
human\textquotesingle s role as orchestrator would remain essential for
setting the strategic direction and providing ethical oversight, but the
"orchestra" would become a single, highly versatile virtuoso.

This trajectory connects directly to the emerging concept of
\textbf{"No-UI"} or "Invisible UI" in enterprise
systems.\cite{sap2025interface} As AI becomes more proactive,
conversational, and context-aware, the need for complex graphical user
interfaces (GUIs) for many tasks may diminish.\cite{sap2025interface}
Interaction will shift from clicking through menus to having a
conversation or allowing an agent to act based on inferred intent. This
suggests that the "Deliver" phase of the framework may, in the future,
focus less on designing screens and more on designing conversations,
agentic behaviors, and robust systems of intent.

Ultimately, the very concept of a fixed, named design process like the
"Double Diamond" or even the "Orchestrated Diamond" may be a
transitional artifact. An advanced agentic AI could one day analyze a
specific design challenge and dynamically construct a bespoke process on
the fly, tailored perfectly to the problem, the team, and the available
resources. It would create its own "diamond." Therefore, the framework
proposed in this paper can be understood as a necessary scaffold---a
structured model to help humans learn the principles of how to think
about and orchestrate human-AI collaboration. Its ultimate success may
be its own obsolescence, once humans have fully internalized these
principles and AI has become capable enough to co-create not just the
solution, but the process itself.

\section{
Conclusion}\label{section-7-conclusion}

\subsection{Summary of
Contributions}\label{summary-of-contributions}

This paper has sought to bridge the growing chasm between established
design methodologies and the transformative capabilities of modern
Artificial Intelligence. In doing so, it has made several key
contributions to the discourse on design theory and practice.

First, it provided a comprehensive critique of the seminal Double
Diamond process, synthesizing disparate criticisms to identify its core
architectural flaw: a rigid, sequential structure that creates a
temporal disconnect between problem-finding and solution-building,
rendering it ill-suited for the iterative and complex demands of
contemporary development.

Second, it formalized the popular but informal "AI Sandwich" metaphor,
grounding it in established Human-Computer Interaction theory and
placing it within a broader \textbf{Spectrum of Agency}. This taxonomy,
which clearly delineates between AI-in-the-Loop, Human-in-the-Loop, and
Mixed-Initiative collaboration models, provides the rigorous theoretical
foundation necessary for building structured, AI-integrated workflows.

Third, based on this foundation, the paper proposed \textbf{The
Orchestrated Diamond}, a novel framework that resolves the Double
Diamond\textquotesingle s deficiencies by reconceptualizing its phases
as distinct modes of human-AI collaboration. This model provides a
practical, phase-specific guide for leveraging the right kind of AI
partnership for the right design task.

Fourth, the framework was validated through a multi-faceted analysis.
Its benefits in overcoming the DD\textquotesingle s flaws were
articulated, its inherent challenges related to bias, data, and skills
were addressed with mitigation strategies, and its principles were
grounded in reality through strong parallels with empirical case studies
across multiple industries.

Finally, the paper offered a forward-looking discussion on the profound
implications of this new paradigm, arguing for an evolution in
terminology from "augmentation" to "orchestration" and outlining the
critical shift in the designer\textquotesingle s role from a hands-on
creator to a strategic orchestrator of a human-AI team.

\subsection{Final Statement: The Future
of Design as Collaborative
Orchestration}\label{final-statement-the-future-of-design-as-collaborative-orchestration}

The integration of Artificial Intelligence into the creative process
does not herald the obsolescence of the human designer; on the contrary,
it elevates the role to one of unprecedented strategic importance. The
future of design innovation lies not in the autonomous capabilities of
AI alone, nor in the unassisted intuition of the human, but in our
ability to skillfully, thoughtfully, and ethically orchestrate a new,
synergistic partnership between human creativity and machine
intelligence. The most valuable products, services, and systems of the
coming decades will be born from this collaboration. The Orchestrated
Diamond provides the first comprehensive map for navigating this
exciting and transformative new landscape, offering a structured path
for design teams to harness the power of AI not just to design things
right, but to discover and develop the right things to design.

% Print bibliography
\printbibliography

% MANUAL BIBLIOGRAPHY CONVERSION NEEDED
% The following manual bibliography entries should be converted to BibTeX format
% and added to references.bib. This is kept here temporarily for reference:
\iffalse
\begin{enumerate}
\def\labelenumi{\arabic{enumi}.}
\item
  AI enters the classroom as law schools prep students for a tech-driven
  practice, accessed July 27, 2025,
  \href{https://nationaljurist.com/prelaw/ai-enters-the-classroom-as-law-schools-prep-students-for-a-tech-driven-practice/}{\ul{https://nationaljurist.com/prelaw/ai-enters-the-classroom-as-law-schools-prep-students-for-a-tech-driven-practice/}}
\item
  How to Use AI to Enhance Engagement and Improve Learning ..., accessed
  July 27, 2025,
  \href{https://www.disco.co/blog/how-to-use-ai-to-enhance-engagement-and-improve-learning-outcomes}{\ul{https://www.disco.co/blog/how-to-use-ai-to-enhance-engagement-and-improve-learning-outcomes}}
\item
  Human-in-the-loop or AI-in-the-loop? Automate or Collaborate? - arXiv,
  accessed July 27, 2025,
  \href{https://arxiv.org/html/2412.14232v1}{\ul{https://arxiv.org/html/2412.14232v1}}
\item
  AI in the Loop vs Human in the Loop: A Technical Analysis of Hybrid
  ..., accessed July 27, 2025,
  \href{https://community.ibm.com/community/user/blogs/anuj-bahuguna/2025/05/25/ai-in-the-loop-vs-human-in-the-loop}{\ul{https://community.ibm.com/community/user/blogs/anuj-bahuguna/2025/05/25/ai-in-the-loop-vs-human-in-the-loop}}
\item
  Integrating Artificial Intelligence into Design Thinking: A
  Comprehensive Examination of the Principles and Potentialities of AI
  for Design Thinking Framework - InfoScience Trends, accessed July 27,
  2025,
  \href{https://www.isjtrend.com/article_199162.html}{\ul{https://www.isjtrend.com/article\_199162.html}}
\item
  The Intersection of Design Thinking and AI: Enhancing Innovation -
  IDEO U, accessed July 27, 2025,
  \href{https://www.ideou.com/blogs/inspiration/ai-and-design-thinking}{\ul{https://www.ideou.com/blogs/inspiration/ai-and-design-thinking}}
\item
  (PDF) Design Thinking and AI: A New Frontier for Designing Human ...,
  accessed July 27, 2025,
  \href{https://www.researchgate.net/publication/362317934_Design_Thinking_and_AI_A_New_Frontier_for_Designing_Human-Centered_AI_Solutions}{\ul{https://www.researchgate.net/publication/362317934\_Design\_Thinking\_and\_AI\_A\_New\_Frontier\_for\_Designing\_Human-Centered\_AI\_Solutions}}
\item
  -Triple diamond framework (source: authors) \textbar{} Download
  Scientific Diagram - ResearchGate, accessed July 27, 2025,
  \href{https://www.researchgate.net/figure/Triple-diamond-framework-source-authors_fig1_347731659}{\ul{https://www.researchgate.net/figure/Triple-diamond-framework-source-authors\_fig1\_347731659}}
\item
  Innovation by Design - Evolving the Double Diamond \textbar{} Equal
  Experts, accessed July 27, 2025,
  \href{https://www.equalexperts.com/blog/our-thinking/innovation-by-design-evolving-double-diamond/}{\ul{https://www.equalexperts.com/blog/our-thinking/innovation-by-design-evolving-double-diamond/}}
\item
  Triple Diamond method for problem solving and design thinking. Rubric
  validation - Semantic Scholar, accessed July 27, 2025,
  \href{https://pdfs.semanticscholar.org/cdf2/3133ec2bb071646dd34cfe038504edc681cd.pdf}{\ul{https://pdfs.semanticscholar.org/cdf2/3133ec2bb071646dd34cfe038504edc681cd.pdf}}
\item
  The double diamond is overrated and breeds unfeasible solutions,
  change my mind - Reddit, accessed July 27, 2025,
  \href{https://www.reddit.com/r/UserExperienceDesign/comments/ts32yw/the_double_diamond_is_overrated_and_breeds/}{\ul{https://www.reddit.com/r/UserExperienceDesign/comments/ts32yw/the\_double\_diamond\_is\_overrated\_and\_breeds/}}
\item
  Double Diamond Is Not How Most Companies Work - Smart Interface Design
  Patterns, accessed July 27, 2025,
  \href{https://smart-interface-design-patterns.com/articles/double-diamond-process/}{\ul{https://smart-interface-design-patterns.com/articles/double-diamond-process/}}
\item
  Why the double diamond isn\textquotesingle t enough \textbar{} by
  adam23gray - UX Collective, accessed July 27, 2025,
  \href{https://uxdesign.cc/why-the-double-diamond-isnt-enough-adaa48a8aec1}{\ul{https://uxdesign.cc/why-the-double-diamond-isnt-enough-adaa48a8aec1}}
\item
  What is AI Orchestration? \textbar{} IBM, accessed July 27, 2025,
  \href{https://www.ibm.com/think/topics/ai-orchestration}{\ul{https://www.ibm.com/think/topics/ai-orchestration}}
\item
  What is AI Orchestration? A Clear Guide for Modern AI Workflows
  \textbar{} Simplified, accessed July 27, 2025,
  \href{https://simplified.com/blog/automation/ai-orchestration}{\ul{https://simplified.com/blog/automation/ai-orchestration}}
\item
  AI in Architectural Design - Real-World Examples ..., accessed July
  27, 2025,
  \href{https://www.architecturecourses.org/ai/ai-architectural-design-real-world-examples}{\ul{https://www.architecturecourses.org/ai/ai-architectural-design-real-world-examples}}
\item
  Maximizing AI Investment: A Guide to Measuring ROI Effectively - SAI
  Digital, accessed July 27, 2025,
  \href{https://www.sai-digital.com/articles/Measuring-the-ROI-of-AI}{\ul{https://www.sai-digital.com/articles/Measuring-the-ROI-of-AI}}
\item
  Trust and UI Design: The Heart of Human-AI Teaming - LMI, accessed
  July 27, 2025,
  \href{https://lmisolutions.com/blog/trust-and-ui-design-heart-human-ai-teaming}{\ul{https://lmisolutions.com/blog/trust-and-ui-design-heart-human-ai-teaming}}
\item
  (PDF) Human-AI collaboration by design - ResearchGate, accessed July
  27, 2025,
  \href{https://www.researchgate.net/publication/380658063_Human-AI_collaboration_by_design}{\ul{https://www.researchgate.net/publication/380658063\_Human-AI\_collaboration\_by\_design}}
\item
  Looking ahead: The future of AI-driven workflows - IT Brief Australia,
  accessed July 27, 2025,
  \href{https://itbrief.com.au/story/looking-ahead-the-future-of-ai-driven-workflows}{\ul{https://itbrief.com.au/story/looking-ahead-the-future-of-ai-driven-workflows}}
\item
  Forrester Report: Agentic AI Turns Generative AI Words into Action
  ..., accessed July 27, 2025,
  \href{https://www.nice.com/resources/forrester-report-with-agentic-ai-generative-ai-is-evolving-from-words-to-actions}{\ul{https://www.nice.com/resources/forrester-report-with-agentic-ai-generative-ai-is-evolving-from-words-to-actions}}
\item
  Where did the interface go - SAP, accessed July 27, 2025,
  \href{https://www.sap.com/design/stories-resources/where-did-the-interface-go}{\ul{https://www.sap.com/design/stories-resources/where-did-the-interface-go}}
\item
  The Rise of AI and Automation in Enterprise System Functionality ...,
  accessed July 27, 2025,
  \href{https://cogleus.com/the-rise-of-ai-and-automation-in-enterprise-system-functionality-enhancing-operations-for-the-future/}{\ul{https://cogleus.com/the-rise-of-ai-and-automation-in-enterprise-system-functionality-enhancing-operations-for-the-future/}}
\item
  (PDF) The Evolution of Automation in IT Operations: From Basic Scripts
  to AI-Powered Solutions - ResearchGate, accessed July 27, 2025,
  \href{https://www.researchgate.net/publication/388910312_The_Evolution_of_Automation_in_IT_Operations_From_Basic_Scripts_to_AI-Powered_Solutions}{\ul{https://www.researchgate.net/publication/388910312\_The\_Evolution\_of\_Automation\_in\_IT\_Operations\_From\_Basic\_Scripts\_to\_AI-Powered\_Solutions}}
\item
  How Intelligent Process Automation is Revolutionizing Enterprise
  Operations, accessed July 27, 2025,
  \href{https://hexaware.com/blogs/intelligent-process-automation-revolutionizing-enterprise-operations/}{\ul{https://hexaware.com/blogs/intelligent-process-automation-revolutionizing-enterprise-operations/}}
\item
  The state of AI: How organizations are rewiring to capture value -
  McKinsey, accessed July 27, 2025,
  \href{https://www.mckinsey.com/capabilities/quantumblack/our-insights/the-state-of-ai}{\ul{https://www.mckinsey.com/capabilities/quantumblack/our-insights/the-state-of-ai}}
\item
  Generative AI: What Is It, Tools, Models, Applications and Use Cases,
  accessed July 27, 2025,
  \href{https://www.gartner.com/en/topics/generative-ai}{\ul{https://www.gartner.com/en/topics/generative-ai}}
\item
  en.wikipedia.org, accessed July 27, 2025,
  \href{https://en.wikipedia.org/wiki/Double_Diamond_(design_process_model)\#:~:text=Double\%20Diamond\%20is\%20the\%20name,B\%C3\%A1n\%C3\%A1thy.}{\ul{https://en.wikipedia.org/wiki/Double\_Diamond\_(design\_process\_model)\#:\textasciitilde:text=Double\%20Diamond\%20is\%20the\%20name,B\%C3\%A1n\%C3\%A1thy.}}
\item
  Double Diamond (design process model) - Wikipedia, accessed July 27,
  2025,
  \href{https://en.wikipedia.org/wiki/Double_Diamond_(design_process_model)}{\ul{https://en.wikipedia.org/wiki/Double\_Diamond\_(design\_process\_model)}}
\item
  History of the Double Diamond - Design Council, accessed July 27,
  2025,
  \href{https://www.designcouncil.org.uk/our-resources/the-double-diamond/history-of-the-double-diamond/}{\ul{https://www.designcouncil.org.uk/our-resources/the-double-diamond/history-of-the-double-diamond/}}
\item
  The Double Diamond design process --- still fit for purpose?
  \textbar{} by ..., accessed July 27, 2025,
  \href{https://medium.com/design-council/the-double-diamond-design-process-still-fit-for-purpose-fc619bbd2ad3}{\ul{https://medium.com/design-council/the-double-diamond-design-process-still-fit-for-purpose-fc619bbd2ad3}}
\item
  Symbiotic Intelligence: The Future Where Humans and AI Unite
  \textbar{} by Eddi Weinwurm, accessed July 27, 2025,
  \href{https://medium.com/@weinwurm/symbiotic-intelligence-the-future-where-humans-and-ai-unite-0e64eaec6ea1}{\ul{https://medium.com/@weinwurm/symbiotic-intelligence-the-future-where-humans-and-ai-unite-0e64eaec6ea1}}
\item
  The Double Diamond Process: From Problems to Solutions \textbar{}
  Maze, accessed July 27, 2025,
  \href{https://maze.co/blog/double-diamond-design-process/}{\ul{https://maze.co/blog/double-diamond-design-process/}}
\item
  The limitations of the Double Diamond - Dan Ramsden - Design leader,
  professional coach, information architecture specialist, magician,
  accessed July 27, 2025,
  \href{https://danramsden.com/2023/07/26/the-limitations-of-the-double-diamond/}{\ul{https://danramsden.com/2023/07/26/the-limitations-of-the-double-diamond/}}
\item
  The Double Diamond method: its history and current uses - Klaxoon,
  accessed July 27, 2025,
  \href{https://klaxoon.com/insight/the-double-diamond-method-its-history-and-current-uses}{\ul{https://klaxoon.com/insight/the-double-diamond-method-its-history-and-current-uses}}
\item
  What Exactly Is Product Design - Eleken, accessed July 27, 2025,
  \href{https://www.eleken.co/blog-posts/what-is-product-design}{\ul{https://www.eleken.co/blog-posts/what-is-product-design}}
\item
  Model Context Protocol: Taming the AI Beast Without a PhD - Johnny
  Bilotta, accessed July 27, 2025,
  \href{https://johnnybilotta.com/post/250515-mcp-taming-the-ai-beast/}{\ul{https://johnnybilotta.com/post/250515-mcp-taming-the-ai-beast/}}
\item
  What is Human-in-the-Loop (HITL) in AI \& ML - Google Cloud, accessed
  July 27, 2025,
  \href{https://cloud.google.com/discover/human-in-the-loop}{\ul{https://cloud.google.com/discover/human-in-the-loop}}
\item
  Putting a human in the loop: Increasing uptake, but decreasing
  accuracy of automated decision-making - PMC, accessed July 27, 2025,
  \href{https://pmc.ncbi.nlm.nih.gov/articles/PMC10857587/}{\ul{https://pmc.ncbi.nlm.nih.gov/articles/PMC10857587/}}
\item
  Mixed-Initiative Systems for Collaborative Problem Solving \textbar{}
  AI Magazine, accessed July 27, 2025,
  \href{https://ojs.aaai.org/aimagazine/index.php/aimagazine/article/view/2037}{\ul{https://ojs.aaai.org/aimagazine/index.php/aimagazine/article/view/2037}}
\item
  Boosting Mixed-Initiative Co-Creativity in Game Design: A Tutorial -
  ResearchGate, accessed July 27, 2025,
  \href{https://www.researchgate.net/publication/377435724_Boosting_Mixed-Initiative_Co-Creativity_in_Game_Design_A_Tutorial}{\ul{https://www.researchgate.net/publication/377435724\_Boosting\_Mixed-Initiative\_Co-Creativity\_in\_Game\_Design\_A\_Tutorial}}
\item
  On Mixed-Initiative Content Creation for Video Games - Goldsmiths
  Research Online, accessed July 27, 2025,
  \href{https://research.gold.ac.uk/id/eprint/38005/1/On_Mixed-Initiative_Content_Creation_for_Video_Games.pdf}{\ul{https://research.gold.ac.uk/id/eprint/38005/1/On\_Mixed-Initiative\_Content\_Creation\_for\_Video\_Games.pdf}}
\item
  Beyond Following: Mixing Active Initiative into Computational
  Creativity - CEUR-WS.org, accessed July 27, 2025,
  \href{https://ceur-ws.org/Vol-3926/paper6.pdf}{\ul{https://ceur-ws.org/Vol-3926/paper6.pdf}}
\item
  Design implications for Designing with a Collaborative AI -
  ResearchGate, accessed July 27, 2025,
  \href{https://www.researchgate.net/publication/318900744_Design_implications_for_Designing_with_a_Collaborative_AI}{\ul{https://www.researchgate.net/publication/318900744\_Design\_implications\_for\_Designing\_with\_a\_Collaborative\_AI}}
\item
  Defining human-AI teaming the human-centered way: a scoping review and
  network analysis - PMC - PubMed Central, accessed July 27, 2025,
  \href{https://pmc.ncbi.nlm.nih.gov/articles/PMC10570436/}{\ul{https://pmc.ncbi.nlm.nih.gov/articles/PMC10570436/}}
\item
  Collaborative Intelligence Explained: How Humans and AI Work Smarter
  Together, accessed July 27, 2025,
  \href{https://www.automationanywhere.com/company/blog/automation-ai/collaborative-intelligence-explained-how-humans-and-ai-work-smarter}{\ul{https://www.automationanywhere.com/company/blog/automation-ai/collaborative-intelligence-explained-how-humans-and-ai-work-smarter}}
\item
  Full article: Collaborative Intelligence: A Scoping Review Of Current
  Applications, accessed July 27, 2025,
  \href{https://www.tandfonline.com/doi/full/10.1080/08839514.2024.2327890}{\ul{https://www.tandfonline.com/doi/full/10.1080/08839514.2024.2327890}}
\item
  Exploring Human-AI Collaboration in Creative Industries - SmythOS,
  accessed July 27, 2025,
  \href{https://smythos.com/ai-trends/human-ai-collaboration-in-creative-industries/}{\ul{https://smythos.com/ai-trends/human-ai-collaboration-in-creative-industries/}}
\item
  Collaborative Intelligence \textbar{} How Humans \& AI Work Together?
  - InvoZone, accessed July 27, 2025,
  \href{https://invozone.com/blog/collaborative-intelligence/}{\ul{https://invozone.com/blog/collaborative-intelligence/}}
\item
  Navigating the Era of Collaborative Intelligence: Humans and AI in
  Concert - Medium, accessed July 27, 2025,
  \href{https://medium.com/@techmsy/navigating-the-era-of-collaborative-intelligence-humans-and-ai-in-concert-e95cc65070a8}{\ul{https://medium.com/@techmsy/navigating-the-era-of-collaborative-intelligence-humans-and-ai-in-concert-e95cc65070a8}}
\item
  Conversational UI: 6 Best Practices in 2025 - AIMultiple, accessed
  July 27, 2025,
  \href{https://research.aimultiple.com/conversational-ui/}{\ul{https://research.aimultiple.com/conversational-ui/}}
\item
  Combining Design Thinking and Generative AI Technologies in The
  Classroom: A Project-Based Learning Approach - Manifold @CUNY,
  accessed July 27, 2025,
  \href{https://cuny.manifoldapp.org/read/combining-design-thinking-and-generative-ai-technologies-in-the-classroom/section/52735f7f-8a08-4690-8926-a08ba5e8cf73}{\ul{https://cuny.manifoldapp.org/read/combining-design-thinking-and-generative-ai-technologies-in-the-classroom/section/52735f7f-8a08-4690-8926-a08ba5e8cf73}}
\item
  AI Workflow - IBM, accessed July 27, 2025,
  \href{https://www.ibm.com/think/topics/ai-workflow}{\ul{https://www.ibm.com/think/topics/ai-workflow}}
\item
  The Impact of AI on Team Development \textbar{} IPM, accessed July 27,
  2025,
  \href{https://instituteprojectmanagement.com/blog/the-impact-of-ai-on-team-development/}{\ul{https://instituteprojectmanagement.com/blog/the-impact-of-ai-on-team-development/}}
\item
  ROI of AI: Key Drivers, KPIs \& Challenges \textbar{} DataCamp,
  accessed July 27, 2025,
  \href{https://www.datacamp.com/blog/roi-of-ai}{\ul{https://www.datacamp.com/blog/roi-of-ai}}
\item
  (PDF) Design Thinking with AI - ResearchGate, accessed July 27, 2025,
  \href{https://www.researchgate.net/publication/381977036_Design_Thinking_with_AI}{\ul{https://www.researchgate.net/publication/381977036\_Design\_Thinking\_with\_AI}}
\item
  The Impact of Artificial Intelligence on Design: Enhancing Creativity
  and Efficiency, accessed July 27, 2025,
  \href{https://www.researchgate.net/publication/384953179_The_Impact_of_Artificial_Intelligence_on_Design_Enhancing_Creativity_and_Efficiency}{\ul{https://www.researchgate.net/publication/384953179\_The\_Impact\_of\_Artificial\_Intelligence\_on\_Design\_Enhancing\_Creativity\_and\_Efficiency}}
\item
  Design GenAI-Powered Experiences Responsibly - Forrester, accessed
  July 27, 2025,
  \href{https://www.forrester.com/blogs/design-genai-powered-experiences-responsibly/}{\ul{https://www.forrester.com/blogs/design-genai-powered-experiences-responsibly/}}
\item
  Here\textquotesingle s how designers are using AI in product design -
  LogRocket Blog, accessed July 27, 2025,
  \href{https://blog.logrocket.com/ux-design/how-designers-use-ai-product-design/}{\ul{https://blog.logrocket.com/ux-design/how-designers-use-ai-product-design/}}
\item
  Developing a design thinking artificial intelligence driven
  auto-marking/grading system for assessments to reduce the workload o -
  Frontiers, accessed July 27, 2025,
  \href{https://www.frontiersin.org/journals/education/articles/10.3389/feduc.2024.1512569/pdf}{\ul{https://www.frontiersin.org/journals/education/articles/10.3389/feduc.2024.1512569/pdf}}
\item
  How AI Is Transforming Sustainable Architectural Design In The USA -
  The AEC Associates, accessed July 27, 2025,
  \href{https://theaecassociates.com/blog/ai-sustainable-architectural-design/}{\ul{https://theaecassociates.com/blog/ai-sustainable-architectural-design/}}
\item
  Ethics of Artificial Intelligence \textbar{} UNESCO, accessed July 27,
  2025,
  \href{https://www.unesco.org/en/artificial-intelligence/recommendation-ethics}{\ul{https://www.unesco.org/en/artificial-intelligence/recommendation-ethics}}
\item
  Intersection between Design Thinking \& AI Thinking - Onething Design,
  accessed July 27, 2025,
  \href{https://www.onething.design/post/ai-thinking-in-design}{\ul{https://www.onething.design/post/ai-thinking-in-design}}
\item
  The Role of AI In Product Development: A Detailed Analysis, accessed
  July 27, 2025,
  \href{https://www.mindinventory.com/blog/ai-in-product-development/}{\ul{https://www.mindinventory.com/blog/ai-in-product-development/}}
\item
  Case Studies: Successful Customer Experience (CX) with AI
  Implementation, accessed July 27, 2025,
  \href{https://www.renascence.io/journal/case-studies-successful-customer-experience-cx-with-ai-implementation}{\ul{https://www.renascence.io/journal/case-studies-successful-customer-experience-cx-with-ai-implementation}}
\item
  5 AI Case Studies in Education \textbar{} VKTR, accessed July 27,
  2025,
  \href{https://www.vktr.com/ai-disruption/5-ai-case-studies-in-education/}{\ul{https://www.vktr.com/ai-disruption/5-ai-case-studies-in-education/}}
\item
  6 AI Use Cases in Educational Content Creation - Ease Learning,
  accessed July 27, 2025,
  \href{https://easelearning.com/all-posts/6-ai-use-cases-in-educational-content-creation/}{\ul{https://easelearning.com/all-posts/6-ai-use-cases-in-educational-content-creation/}}
\item
  How Effective is AI in Education? 10 Case Studies and Examples - Axon
  Park, accessed July 27, 2025,
  \href{https://axonpark.com/how-effective-is-ai-in-education-10-case-studies-and-examples/}{\ul{https://axonpark.com/how-effective-is-ai-in-education-10-case-studies-and-examples/}}
\item
  AI Workflow UI: Design, Management \& Automation \textbar{} Fuselab
  Creative, accessed July 27, 2025,
  \href{https://fuselabcreative.com/ai-workflow-ui-design-management-automation/}{\ul{https://fuselabcreative.com/ai-workflow-ui-design-management-automation/}}
\item
  Human-AI Teaming: Definition, Strategies, and More \textbar{} CO- by
  US Chamber of Commerce, accessed July 27, 2025,
  \href{https://www.uschamber.com/co/run/technology/human-ai-teaming}{\ul{https://www.uschamber.com/co/run/technology/human-ai-teaming}}
\item
  Introducing ChatGPT agent: bridging research and action - OpenAI,
  accessed July 27, 2025,
  \href{https://openai.com/index/introducing-chatgpt-agent/}{\ul{https://openai.com/index/introducing-chatgpt-agent/}}
\item
  Agentic Workflows: How Autonomous AI Executes Complex Tasks - Triple
  Whale, accessed July 27, 2025,
  \href{https://www.triplewhale.com/blog/agentic-workflows}{\ul{https://www.triplewhale.com/blog/agentic-workflows}}
\item
  No UI is the New UI : Towards Future Transparent Interactions -
  NASSCOM Community, accessed July 27, 2025,
  \href{https://community.nasscom.in/communities/emerging-tech/no-ui-new-ui-towards-future-transparent-interactions}{\ul{https://community.nasscom.in/communities/emerging-tech/no-ui-new-ui-towards-future-transparent-interactions}}
\item
  The end of UI as we know it. How AI is shaping a future beyond\ldots{}
  \textbar{} by Remco Snijders \textbar{} Bootcamp \textbar{} Medium,
  accessed July 27, 2025,
  \href{https://medium.com/design-bootcamp/the-end-of-ui-as-we-know-it-c5ec6e983aa1}{\ul{https://medium.com/design-bootcamp/the-end-of-ui-as-we-know-it-c5ec6e983aa1}}
\end{enumerate}
\fi

\end{document}
